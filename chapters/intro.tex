% !TeX root = ../main.tex

\chapter{简介}

物理学问题通常有两种研究范式:
  还原论
  
  演生论是研究大量微观粒子组成的系统演化规律。
    “More is different!"
  
  “微观”、“介观”、“宏观”、“非平衡”、“非线性”、“多尺度”、“跨尺度”、“复杂物理场”
   统计物理学是联系微观与宏观的桥梁。

   多尺度(时间、空间)系统演化

\begin{eqnarray}
	m_a~,\bfj
\end{eqnarray}

  
  
  等离子体具有显著的跨尺度效应。
  
  随着参数的改变与研究的深入,等离子体往往呈现出人意料的物理特征、演化机制以及系统规律

  等离子体是由大量微观带电粒子组成的宏观物理系统;通常由于边界条件以及自身演化,等离子体系统存在复杂电磁场。等离子体系统演化本质上可看作带电粒子间相互作用(碰撞效应)以及带电粒子与电磁场相互作用(平均场效应)两方面。通常,等离子体系统处于热力学非平衡态。根据等离子体的离散程度或者非平衡度可分为四个区间()
  
  根据研究所研究等离子体系统的时间与空间尺度,通常有微观、介观与宏观三种层次的模型及模拟方法。微观层面,当研究对象的相对离散很强时,基于第一性原理的微观粒子描述有单粒子模型等;宏观层面,当等离子体系统处于近热力学平衡态时,基于电磁流体力学连续性假设有理想磁流体模型与双流体模型等;当系统偏离热力学平衡态较远又难以用第一性原理有效描述的区间(又称介观尺度),

  自发系统的非平衡系统演化是个不断趋近于热力学平衡态的过程。克劳修斯熵增原理给出了系统热量输运方向。更一般的动理学矩流是否存在类似地广义熵增原理?
  
  本文基于统计物理学基本研究方法构建跨尺度、自适应物理模型。
  
  当碰撞
  
\section{等离子体物理模型}
\label{等离子体物理模型}

  
  
\subsection{单粒子模型}
\label{单粒子模型}


\subsection{流体模型}
\label{流体模型}

\subsubsection{磁流体}
\label{磁流体}

\subsubsection{双流体}
\label{双流体}


\subsection{动理学模型}
\label{动理学模型}

    直接对原始动理学模型进行理论分析与数值求解,由于其非线性以及其高维空间特性使得理论与数值上皆面临巨大挑战。因此根据所研究的问题的特性对原始动理学方程进行合适的简化,提出了许多近似模型。

    Vlasov-BGK模型

    Coulomb碰撞\cite{Heikkinen2007}
   
\section{非平衡统计热力学}
\label{非平衡统计热力学}

\section{偏微分方程的数值解法}
\label{偏微分方程的数值解法}

偏微分方程广泛存在于物理、天文、力学等自然科学以及工程问题中。然而实际问题中的偏微分方程(组)常常由于其多维性、非线性等难以寻找解析解。随着现代计算能力的不断发展,数值求近似解的方法是偏微分方程的主要求解方法。根据偏微分方程的具体形式以及所关注的系统性质对应发展了不同类型的数值方法。根据局域性、网格依赖性、算法的阶数、显式/隐式等不同特性,数值方法有不同的分类。

根据具体依赖的方程形式可分为强形式算法与弱形式算法。强形式方法是直接基于原始偏微分方程直接进行数值求解。局域方法中有限差分法、全局方法中谱方法等属于典型的强形式方法。 反之,弱形式是根据偏微分方程导出其弱积分形式,然后基于此等价弱形式进行数值求解的方法。

强形式算法通常构造更简单、计算高效;然而弱形式算法通常比强形式具有更友好的数值稳定性与适应性。首先,积分方程中可以通过分部积分以降低方程的最高求导阶数,使得弱形式算法可降低对函数连续性阶数的要求。其次,积分方程在数学上能够降低函数近似时可能引起数值误差,因此弱形式通常可以改善数值方法的稳定性、提高解的精度。加权余量法(Method of Weighted Residuals)是广泛应用的弱形式方法之一。

\subsection{加权余量法}
\label{加权余量法}

加权余量法的核心思想是采用试函数代入弱形式方程,通过在给定权函数时使其残差最小化来确定试函数中待定系数,从而给出原方程的近似数值解。有限元法、无网格方法等都是基于加权余量法的特殊形式。根据权函数的形式,加权余量法可分为配点法(Collection Method)、子域法(Subdomain Method)、Galerkin法以及最小二乘法(Least Square Method)、矩方法(Method of Moment)等。

加权余量法中试函数的建立应遵循以下原则:

\begin{itemize}
\item[I.]试函数由完备函数集的子集构成。常用的试函数包括三角函数、幂级数、样条函数、Legendre函数、Chebyshev函数、Laguerre函数、Bessel函数等。
\item[II.]试函数至少应具有比消除余量的加权积分式中最高阶导数低一阶的函数光滑性。
\item[III.]试函数应该与原问题的解析解或者特解有关联。当试函数包含原问题越多特征,比如对称性、渐进性、极值个数等信息,弱形式算法具有越高的收敛性与计算效率。
\end{itemize}

根据网格依赖性,数值计算方法可分为传统网格类方法以及无网格方法;

\subsection{网格类方法}
\label{网格类方法}

数值计算方法中有限差分法(Finite difference method,FDM)、有限体积法(Finite volume method,FVM)和有限元法(Finite element method,FEM)是最常用的偏微分方程数值计算方法。此三种方法都需要在计算域空间先生成离散网格,属于网格类方法。

\begin{itemize}
\item[1]
有限差分法。在众多数值分析方法中,有限差分法是通过在足够小的时空区域内,用临近离散格点值的特定组合(如线性或者近线性组合)逼近微积分,从而把微积分方程转换成可以用矩阵代数技术求解的线性方程组。有限差格式的数学理论完备,通常代数矩阵具有稀疏特性使得数值求解简单且易于编程实现,在现代数值分析中得到了广泛的应用。此法格式简单,但通常其显式格式稳定性条件较苛刻、隐式格式迭代求解成本较大。
\item[2]
有限体积法。 其基本思想是通过计算节点临近体积元上解的平均值的精确表达式,并使用这些数据来构造单元内解的近似值。此方法需要原方程具有流守恒格式,并利用散度定理将包含散度项的偏微分方程中的体积积分转化为对应的曲面积分,然后将这些项计算为每个体积元表面上的通量积分。因为进入体积元的通量与离开相邻体积的通量相等,所以这种方法能够严格保证系统在平均意义上的离散守恒性。有限体积法的另一个优点算法容易构造,结果有明确的物理含义,并且允许灵活应用非结构化网格。该方法广泛应用于计算流体动力学问题。但该方法对区域适应性较差,通常解的精度也难以提高。
\item[3]
有限元法。有限元法(文献)的思路是将一个连续的大系统划为更小且简单的子单元,并在子单元内选取合适的试函数,并与权函数一起拟合子单元内的原函数,然后把这些离散方程组装成一个更大的方程系统,最后用直接或者迭代求解方法计算此方程。通常有限元方法中试函数会在子单元内引起局域误差,通常被称为残差;而权函数就是投射残差的多项式近似函数。通过构造一个残差函数和权函数的内积积分,并应用变分方法最小化相关误差函数,从而并使积分趋近于零得到最终的解。误差函数最小化的过程通常是通过迭代实现的。应用后验误差估计方法,当近似误差大于给定的容差时,通过自动适应过程优化离散化过程,直到达到给定的要求。有限元方法的优点是胜任复杂几何区域,并被广泛应用于流体力学、结构力学等领域。此法网格适应性强,但受限于插值阶数难以提高求解精度。
\end{itemize}

网格类方法在数值计算各个领域得到了系统的研究并得到了广泛的应用。其本质上是在应用格点临近信息对函数(原函数、积分、偏导数、梯度等)构造特定插值近似格式实现对偏微分方程的离散近似。也正因此有诸多不足:首先,网格类方法的计算结果的质量通常对网格质量依赖性较强,尤其对于高维偏微分方程。网格的数量、形状以及网格随时间更新特性等直接影响数值结果。其次,对于高维(3D-3V)动理学问题中梯度随时间变化而需要不断更新网格时,网格划分通常需要耗费大量时间与精力;尤其当计算区域不规则时,高质量网格的生成将更加困难。第三,网格类方法通常是低阶算法,只能保证方程低阶守恒性。当处理非局域、大梯度(大变形、流体界面等)问题时,偏微分方程具有显著非线性效应(如等离子体中动理学效应);此时网格类方法应用局限性较大。

\subsection{谱方法}
\label{谱方法}

最佳逼近

最佳平方逼近

最佳一致逼近

谱方法(文献)是应用数学中用于数值求解微分方程的一类重要技术。不同于有限差分法等对微分方程本身进行离散,谱方法是对结果做高精度的拟合。其基本思想是把微分方程的解通过写成特定“基函数”的组合近似表示出来,然后计算基函数对应的系数以尽可能满足原微分方程。
\begin{theorem} \label{定理-Cover}
    \textbf{Cover定理}: 复杂模式分类问题非线性的投射到高维空间总是比到低维空间更具有线性可分性。
\end{theorem}

由Cover定理可知,非线性可分问题向线性可分问题变化需要满足两个条件:第一,变换由地位空间向高维空间;第二,变换是非线性的。

\begin{theorem} \label{定理-Micchelli}
    \textbf{Micchelli定理}: 如果$\{x_i, i=1:1:N\}$是$\bbR^n$中$N$个互不相同的点集合,则$N\times N$阶插值矩阵$\boldsymbol{\Phi} = \left[\phi_{j,i}=\phi \left(||\bfx_j - \bfc_i|| \right) \right]_{N\times N}$是非奇异的。
\end{theorem}
径向基函数均满足Micchelli定理。

\begin{theorem} \label{定理-正则化原理}
    \textbf{正则化原理}(Tikhonov,1963): 正则化问题最小解$F(\bfx)$是$N$个Green函数线性叠加。其目标函数为$F(\bfx)=\sum_{i=1}^n w_i G(\bfx;c_i)$。其中$\bfx_i$是中心向量,权值$Fw_i=[d_i - F{\bfx_i}] / \lambda$为展开系数。最小解与系统误差呈线性关系、与正则化参数$\lambda$成反比。
\end{theorem}
以正交基函数为基的谱分解可看做是对原函数构造正则化网络来逼近的过程。对于连续函数,选取足够多的正交展开基可以构成一个偏差泛函最小化的通用最佳逼近器。谱方法和有限元方法都是用基函数来近似数值解。然而,不同于有限元方法使用的基函数仅在邻近的子区域上是非零的,谱方法使用的基函数在整个区域上是非零的。换而言之,谱方法是全局方法,而有限元方法采用局域方法。因此谱方法有很好的误差收敛性质,且有当解是平滑的时候能达到指数收敛。此即意味着,谱方法能够用少量的节点来实现高精度计算。

% \begin{theorem} \label{定理-等价原理}
%     等价原理:对于有限非负整数$p \in [0,\bbN^+$]及整数$l,j \in \bbN$,定义域为$x \in [0,\infty)$的函数$f_l(x)$具有如下渐进性
%     \begin{eqnarray}
%         \DDxp f_l (x \to \infty) & = & 0, \quad  p \ne \infty , \label{DDx9pfl} 
%         \\
%         \DDxp f_l(x=0)  & = & 
%         \begin{aligned}
%         % \begin{split}
%             \left \{
%             \begin{array}{cc}
%                 c_l, &  \quad p = l \\
%                 0,     &  \quad 3 \le p < l \\
%                 c_{l,p}, &  \quad p < l \le 2
%             \end{array}
%             \right .
%         % \end{split}
%         \end{aligned}
%         ~. \label{DDx0pfl}
%     \end{eqnarray}
%     其中,$c_l$、$c_{l,p}$是与$x$无关的常数。若函数$g_l(x)$在边界点处满足
%     \begin{eqnarray}
%         \DDxp g_l(x \to \infty)  & = & 0 , \quad p  \le l ,  \label{DDx9pgl}
%         \\
%         \DDxp g_l(0) & = & 0, \quad 3 \le p < l  \label{DDx0pgl}
%     \end{eqnarray}
%     并且在全域满足
%     \begin{eqnarray}
%         \left | \int_0^\infty x^{j+2} g(x) \rmd x - \int_0^\infty x^{j+2} f (x) \rmd x \right |  & \le & eps , \quad - (l+2) \le j_m \le j \le j_M ,
%     \end{eqnarray}
%     则函数$g(x)$与原函数$f(x)$具有$(j_m,j_M)$阶弱等价性。
% \end{theorem}

\begin{theorem} \label{定理-等价原理}
    \textbf{等价原理}:定义域为$x \in [0,\infty)$的连续函数$y(x)$具有如下渐进性
    \begin{eqnarray}
        \DDxp y(x=0)  & = & c_p^0, \label{DDx0py}
        \\
        \DDxp y (x \to \infty) & = & c_p^\infty, \quad p \in [0,\bbN^+]  ~. \label{DDx9py} 
    \end{eqnarray}
    其中,$c_p^0$与$c_p^\infty$是与$x$无关的常数。函数$Y(x)$的前第$p$阶导数在边界点处为
    \begin{eqnarray}
        \DDxp Y(x=0) & = & C_p^0,  \label{DDx0pY}
        \\
        \DDxp Y(x \to \infty)  & = &  C_p^\infty,  \label{DDx9pY}
    \end{eqnarray}
    其中,常数$C_p^0$与$C_p^\infty$满足
    \begin{eqnarray}
        \left |C_p^0 - c_p^0  \right | & \le & eps, 
        \\
        \left |C_p^\infty - c_p^\infty  \right |  & \le & eps ~.
    \end{eqnarray}
    若函数$Y(x)$的第$j \in \bbN$阶在全域的积分满足
    \begin{eqnarray}
        \left | \int_0^\infty x^{j+2} Y(x) \rmd x - \int_0^\infty x^{j+2} y (x) \rmd x \right |  & \le & eps, \quad j_m \le j \le j_M,
    \end{eqnarray}
    则函数$Y(x)$与原函数$y(x)$具有$(j_m,j_M)$阶弱等价性。
\end{theorem}

\begin{theorem} \label{定理-谱展开}
    对于连续光滑函数$y=y(x), x \in [a,b]$,可用基矢函数$\phi (x)$对其展开为
    \begin{eqnarray}
      y(x)  & = &  \sum_{n=0}^{\infty}  c_n \phi_n (x) ~. \label{ycn}
    \end{eqnarray}
    其中,展开系数$c_n$可由如下形式给出
    \begin{eqnarray}
      c_n  & = &  \int_{a}^{b}  y(x) \phi_n (x) w(x) \rmd x ~. \label{cny}
    \end{eqnarray}
    权重函数$w(x)$满足正交归一性,即
    \begin{eqnarray}
      \int_{a}^{b}  \phi_n (x) \phi_m (x) w(x) \rmd x & = & \delta_m^n ~. \label{wx}
    \end{eqnarray}
\end{theorem}

根据被求解系统的特点可以选取不同基函数,通常有如下规律:
\begin{itemize}
    \item [I.] 如果方程的解具有周期性,则优先选用Fourier级数。
    \item [II.] 如果解不是周期的,求解区域能够通过简单的坐标变化化为正方形或立方体,则优先考虑在每个维度使用Chebyshev多项式。
    \item [III.] 如果求解区域是球面,则角度方向优先选用球谐函数(Legendre多项式与Fourier多项式的积)。半径方向若是有界区域$[r_a,r_b )$(球壳),则应用Chebyshev多项式。
    \item [IV.] 如果求解区域是球体,角度方向依然选用球谐函数。半径方向半无界区域 [0,∞)  ,则采用Laguerre多项式。
    \item [V.] 如果求解区域是轴对称性的,角度方向选用Legendre多项式。
\end{itemize}
选用不同的基矢以及计算系数的方法将给出不同具体的谱方法。


谱方法有鲜明的优缺点。对于光滑函数,谱方法具有无比优越性;高精度计算时,通常比有限元方法等具有更高的计算效率。然而由于吉布斯现象(文献),对于具有复杂几何形状和解函数不连续的问题时,谱方法的求解精度有不同程度的降低。

根据残差控制方法,谱方法的实现通常有Tau方法、Galerkin方法与伪谱法(Collocation)三种技术来确定谱系数[53]。当用配点法求解对应系数的时候通常也被称为伪谱法[54];在求解连续空间中高度非线性方程时具有高效且易于编程实现的特点,
Ross(文献)等人在研究多尺度最优控制问题时提出了伪谱最优控制理论。该理论指出伪谱方法以指数型快速收敛,即使解的分量是高频的,也可以在节点数很少的情况下得到解的逐点收敛并趋近于理论值。

\subsection{伪谱法}
\label{伪谱法}

伪谱方法(文献),又称为离散变量表示法,是求解非线性偏微分方程的重要数值方法。伪谱法的一大优点是应用于非线性问题时易于构造及编程实现。不同于谱方法中的伽辽金法与Tau法,伪谱法是通过额外的伪谱基(通常是高斯节点)来完善基矢,同时允许在正交网格上近似表示函数。在数学上,对于任意光滑函数$(C^{\infty})$,在Gaussian节点上的插值多项式在$L^2$空间以所谓的谱速率快速收敛,这通常快于任何有限多项式的收敛速率。简单来说,谱分析方法在谱空间求解方程,而伪谱法是物理空间中的处理方案。因此,伪谱法本质上等价于全域有限差分法。伪谱法能够高效处理正交网格、并简化了谱方法中某些运算符的计算,提高计算速度。处理特定问题时伪谱法能够使用快速算法(如快速傅里叶变换等)时可以大大加快计算速度。

\begin{theorem} \label{定理-伪谱法}
    对于连续光滑函数$y=y(x), x \in [a,b]$,当$x$采用Gaussian格点$[{x_i}]$离散时,用基矢函数$\phi (x)$展开并在 $N$ 阶截断的离散变量形式为
    \begin{eqnarray}
      y(x_i)  & = &  \sum_{n=0}^{N}  c_n \phi_n (x_i), \quad i = 1:N+1 ~. \label{ycnxi}
    \end{eqnarray}
    其中,${x_i }, i = 1:N+1$ 是基矢函数的前 $N+1$  个零点。通过对方程\EQ{ycn}做逆变换可得开展系数$c_n$,并用Gaussian积分,记为
    \begin{eqnarray}
      c_n  & = &  \left <\phi(x_n) | y(x_i) \right > _G = \sum_{i=1}^{N+1} w_i \phi_n (x_i) y(x_i) ~. \label{cnyG}
    \end{eqnarray}
    其中,下标 “G”代表Gaussian积分;$x_i$与系数$w_i$分别代表第$i$个配置点及其对应的权重。权重$w_i$通常可由文献[70]中的算法给出。上述方程写为矩阵形式为
    \begin{eqnarray}
      \bfc & = & \bsM \cdot \bfy  ~. \label{bfc}
    \end{eqnarray}
    式中,
    \begin{eqnarray}
      \quad M_{n,i} & = & w_i \phi_n (x_i)  ~. \label{Mni}
    \end{eqnarray}
    相应的,方程\EQ{ycnxi}可以表示为
    \begin{eqnarray}
      \bfy & = & \bsM^{-1} \cdot \bfc ~. \label{y}
    \end{eqnarray}
    其中,
    \begin{eqnarray}
      M_{n,i}^{-1} & = & \phi_i (x_n) ~. \label{Mniinv}
    \end{eqnarray}
    由于Gaussian积分对于光滑函数同样有指数收敛速率,通常我们可以以很高的精度计算系数。
\end{theorem}
\noindent


\begin{theorem} \label{定理-球谐函数展开}
  球面上任意\textbf{平方可积函数}可以用球谐函数展开。采用球谐函数为基矢,在速度空间对归一化分布函数做正交展开并保留前$l_M$阶,有
  \begin{eqnarray}
      \fh \left(\r,\vh,t \right) &=& \sumloq \sumlnl \fhlm \left(\r, \rmvh ,t \right) \Ylm \left(\mu, \phi \right) ~. \label{fhsph}
  \end{eqnarray}
   其中,$(\rmvh$、$\mu$、$\phi )$是归一化速度空间的球极坐标空间以及归一化速率$\rmvh$表征到归一化速度空间原点的欧几里得距离(Euclidean norm,L2)$||\vh||$。球谐函数$\Ylm \left(\mu, \phi \right) = \Plmabs  \left(\mu \right) e^{i m \phi}$。本文中球谐函数形式采用无归一化系数$\sqrt{2} N_l^m$的复数形式的。系数函数$\fhlm$是与速度空间角度不相关的复数函数,表征归一化分布函数在速度空间的第$(l,m)$阶振幅。此时,分布函数速度空间的两个角度方向转化为由谱空间$(l,m)$;归一化分布函数由一组$(l,m)$谱空间中振幅函数$\fhlm\left(\r, \rmvh ,t \right)$共同描述。通过对上述方程做逆变换,当$m\ge0$时振幅函数为
   \begin{eqnarray}
       \fhlm \left(\r,\rmvh,t \right) =  \int_{-1}^1 \int_0^{2 \pi} \frac{1}{2} \left (Y_l^{\left |m \right|} \right)^* \fh \left (\r, \vh, t \right) \rmd \phi  \rmd \mu ~. \label{fhlm}
   \end{eqnarray}
   且满足
   \begin{equation}
       \left(\fhlm \right)^* \ = \ \fh_{l}^{-m} ~. \label{flmconj}
   \end{equation}
\end{theorem}

应用Gaussian积分,第$(l,m)$阶振幅函数可表示为
    \begin{eqnarray}
      \fhlm \left(\r,\rmvh_i,t \right)  & = & \sum_{j=1}^{l_M+1} \sum_{k=1}^{2l_M+1} w_j^l P_l^m (\mu_j) w_k^m e ^ {-i m \phi_k } \fh \left (\r, \rmvh_i,\mu_j,\phi_k, t \right) ~. \label{flmconjG}
    \end{eqnarray}
权重$w_j^l$与$w_k^m$可由文献[70]中的算法给出。
% 其矩阵形式可表示为
%     \begin{eqnarray}
%       \fhlm \left(\r,\rmvh_i,t \right)  & = & \sum_{j=1}^{l_M+1} \sum_{k=1}^{2l_M+1} w_j^l P_l^m (\mu_j) w_k^m e ^ {-i m \phi_k } \fh \left (\r, \rmvh_i,\mu_j,\phi_k, t \right) ~. \label{flmconjGM}
%     \end{eqnarray}
% 1
%     \begin{eqnarray}
%       \bffh & = & {\bsM}_\mu \cdot \bffh \cdot {\bsM}_\phi  ~. \label{bffh}
%     \end{eqnarray}
%     式中,
%     \begin{eqnarray}
%       {\quad M_\mu}_{l,j} & = & w_j^l P_l^m (\mu_j), \label{Mmulj}
%       \\
%       {\quad M_\phi}_{m,k} & = & w_k^m e ^ {-i m \phi_k }  ~. \label{Mphimk}
%     \end{eqnarray}
为应用方便,记复函数
   \begin{equation}
       Y \ = \ Y_{\R} + i Y_{\I} , \label{YRI}
   \end{equation}
其中,$Y_{\R}$与$Y_{\I}$分别是复函数$Y$的实部与虚部。当地局域坐标空间采用笛卡尔坐标系 $\r=\r \left(x,y,z\right)$,归一化速度空间中球极坐标与笛卡尔坐标$(\rmvh_x$、$\rmvh_y$、$\rmvh_z )$的关系由方程\EQ{vhxyz}-\EQ{vhz}给出。

伪谱法具有插值函数的形式,并且对于光滑函数有当 N→∞ 时,其值与原函数的误差将以指数速率收敛于零。


伪谱法的另一个优势在于可以高效、高精度计算光滑函数的任意阶导数值。例如求一阶导数$\partial_x y(x)$时,把方程\EQ{ycnxi}代入其中求得。由于伪谱法中系数$c_n$是实空间中的展开系数,与自变量无关$x$;因此可以交换求导与连加符号的顺序,从而计算原函数的导数。

\begin{corollary} \label{推论-微分矩阵}
    对于连续光滑函数$y=y(x), x \in [a,b]$,当$x$采用Gaussian配置点$[{x_i}]$离散时,其配置点上的导数值可采用微分矩阵计算,有
    \begin{eqnarray}
      \ddx y(x_i)  & = &  \sum_{n=0}^{\infty} c_n \ddx \phi_n (x_i), \quad i = 1:N+1 ~. \label{ddfxi}
    \end{eqnarray}
    令
    \begin{eqnarray}
      \ddx \phi_n (x_i)  & = &  D_{in}^{(1)} , \quad i = 1:N+1 ~. \label{Din}
    \end{eqnarray}
    此即基函数$\phi_n$对应的一阶微分矩阵的第$i$行$n$第列。则方程\EQ{ddfxi}对应的矩阵形式可表示为
    \begin{eqnarray}
      \ddx y([x_i])  & = & \bsD \cdot \bfc, \quad i = 1:N+1 ~. \label{ddfx}
    \end{eqnarray}
    类似地可计算高阶导数。伪谱法对应的任意阶微分矩阵可以根据文献[71][72]中的算法求出。
\end{corollary}
\noindent

Ross(文献)等人在研究多尺度最优控制问题时提出了伪谱最优控制理论。该理论指出伪谱方法以指数型快速收敛,即使解的分量是高频的,也可以在节点数很少的情况下得到解的逐点收敛并趋近于理论值。

\subsection{无网格方法}
\label{无网格方法}

无网格方法(文献)不依赖于网格构建函数近似。无网格方法利用一组分布在求解域及边界上的节点表征方程的域与边界;且无需利用预先定义的节点网格信息进行域离散以及构造未知函数的插值或者函数近似。得益于不依赖于网格(严格意义的方法)或者只需要一定形式的背景网格,无网格方法在大梯度、网格畸变等非线性效应较强的问题中具有显著优势,成为国内外数值计算各领域的研究热点。针对不同应用需求,目前已提出众多具体格式的无网格方法(文献)。不同无网格方法的主要区别在于具体选取试函数的方法。常用的试函数有重构核函数近似、移动最小二乘法、径向基函数法、单位分解法与点插值法等。
根据离散方法,无网格方法通常可分为基于配点法、Galerkin法与Petrov-Galerkin法三大类。通常基于配点法与Petrov-Galerkin法的无网格方法不需要背景网格;而基于Galerkin法的离散方案的无网格方法大部分需要配以背景网格。

无网格方法与有限元法的主要区别在于构造函数近似格式不借助于网格,基于函数逼近近似而非构造插值近似。有限元中积分通通常是在单元内部建立高斯积分点来计算;而无网格方法则应用节点或者背景网格计算积分。与经典加权余量法的区别在于采用了定义在离散节点上的紧支特性函数来构造函数逼近格式,而非定义在全域上的级数展开形式。无网格方法大多具有如下优点:

\begin{itemize}
\item[(1)] 无网格方法的函数逼近近似过程对网格没有依赖,适合处理大形变、大梯度且随时间变化等非线性效应较强时需要自适应网格/节点的应用问题。
\end{itemize}

\begin{itemize}
\item[(2)] 无网格方法的基函数通常包含能够反应系统特性的函数簇,适用于研究具有大梯度及奇异性等独特性质的问题。
\end{itemize}
 
\begin{itemize}
\item[(3)] 无网格方法前处理只需要节点位置信息而无需网格信息,且算法自适应性强。h自适应性(h-adaptive)无需重新划分网格;p自适应性亦容易实现。当应用紧支函数格式时,无网格方法与有限元法一样具有带状稀疏特性。因此无网格方法适用于分析带复杂三维结构的大型科学与工程问题。
\end{itemize}

近年来发展起来的神经网络、基于物理信息的神经网络方法等属于无网格方法。

Dempster等人于1977年提出的一种面向含有“隐变量”概率模型参数似然估计方法--\textbf{期望最大化算法}(Expectation Maximization,EM)。

\section{常用等离子体数值模拟方法}
\label{常用等离子体数值模拟方法}

  Coulomb碰撞在高密度聚变等离子体(如ICF)中通常有显著的物理效应\cite{Ricketson2014}。在低密度聚变等离子体(MCF)中,Coulomb碰撞通常被认为是可忽略的。然而近期研究表明在Tokamak边界等离子体中Coulomb碰撞依然有不可忽略的效应\cite{Park2010,Koh2012}。即使在通常所谓无碰撞区域的高温等离子体,Coulomb碰撞也是描述小尺度精细结构发展出湍流系统所不可忽略的效应。研究表明,碰撞效应在等离子体湍流输运中有至关重要的作用\cite{Garbet2010}。如离子-离子碰撞引起带状流的衰减,从而增强非线性区离子热输运\cite{Lin1999,Falchetto2004};电子-离子碰撞通过降低捕获粒子驱动力影响捕获电子模(trapped electron mode, TEM)等\cite{Rewoldt1990}。

  回旋动理学(gyrokinetic)\cite{Garbet2010}通过坐标变换消除带电粒子做快速回旋运动的时间尺度,把3D3V维VFP方程转化为3D2V维gyro-VFP方程。回旋动理学通过降维实现降低对动理学方程的求解难度;然而回旋变换基于带电粒子轨迹具有Hamiltonian特征假设\cite{Garbet2010},且Maxwell方程需做对应处理。由于复杂的几何与边界条件,数值求解3D2V维回旋动理学方程依然巨大的挑战。
  
\subsubsection{动理学数值计算方法}
\label{动理学数值计算方法}

  turbulence\cite{Doyle2007} and transport 
  数值直接求解VFP方程通常有动理学宏观建模与直接求解法两种思路。直接数值求解VFP方程常用的有有限粒子方法(finite particle methods)及网格离散方法,包括有限差分法(finite difference methods, FDM)、有限体积法(finite volume methods, FDM)等.
  
  数值求解六维含时VFP方程由于速度空间快速增加的网格量,通常对于计算资源是一个巨大的挑战\cite{Heikkinen2007}。然而,分布函数在速度空间中精细的方位角信息通常很快被碰撞效应光滑掉。一种有效的方式是在速度空间用球谐函数为基对分布函数做展开。这种处理从而减少速度空间角度方向的网格数量。

  物理空间与速度空间可以独立采用不同数值处理方案,(FDM、FVM、粒子采用、伪谱法等)。物理空间离散有许多成熟有效的算法,下面我们主要回顾速度空间数值算法。
  
  求解Vlasov-Fokker-Planck系统通常有两种图景:第一,系综平均场(ensemble-averaged filelds,EAF)模型描述等离子体系统的集体性质;第二,以大量有限粒子(finite-size particles, FSP)的单个系宗为对象研究等离子体的演化。

  求解Maxwell-Vlasov-Fokker-Planck系统有多种途径:
	将分布函数 f(r,v) 用离散粒子的坐标与速度来描述,并通过第一性原理来计算系统的演化。这种方法被称为 Particle-in-Cell (PIC),即粒子在场的网格中运动。此方法物理图像清晰,且容易保持系统的守恒性质。然而此方法不仅数值噪声高、计算量大,需要考虑碰撞效应时通常不容易处理。
	从Landau形式的Maxwell-Vlasov-Fokker-Planck方程出发,将 f(r,v) 完全在6维相空间 (r,v) 中使用有限体积法直接离散,再使用离散后的演化方程来计算 f(r,v) 以得到分布函数的演化。这样做的好处是之一数值噪声低且容易构造满足离散守恒的格式;第二,保留了速度空间的非线性项,能够更好的描述动理学效应。然而由于直接在6维相空间中离散,程序计算量极大。
 
	速度空间首先做线性化处理,如球谐函数(或者笛卡尔张量积)展开后线性化;甚是使用更简单的BGK碰撞频率模型。然后物理空间再应用有限体积法等。这样做的优势是可以显著降低速度空间计算量;然而代价却是可能难以捕捉速度空间非线性效应引起的物理现象。
	当系统扰动不是很大时,δf 方法是一种高效的算法。该方法在一个已知的简单背景分布上叠加一个通过采样获得的扰动分布,从而能够降低数值噪声、减小计算量。然而当扰动较大时,该方法的应用有待考究。

 

在本章,我们首先对Vlasov-Fokker-Planck方程在速度空间进行球谐函数展开,并应用自适应增强型伪谱法在速度空间进行离散。物理空间中选用直角坐标系,应用有限体积法进行构造离散守恒方程,为下一步研究工作打下基础。

\subsubsection{粒子模拟}
\label{粒子模拟}

\subsubsection{有限差分格式 VFP 方程}
\label{有限差分格式 VFP 方程}

  非守恒的有限差分法求解 Fokker-Planck 方程
  由于 Fokker-Planck 碰装内在具有三维特性,使得构造具有守恒性质的有限差分格式成为巨大挑战。然而,碰撞效应总是使得分布函数趋向于各向同性。因此,在速度空间采用球谐函数为基对分布函数进行展开,理论上三维 Fokker-Planck 方程离散为一组具有无穷多个一维 Fokker-Planck 方程组成的方程组。由于碰撞效应总是趋向于减弱各向异性,当系统中没有极端各向异性(强流组分)时,通常可以取方程组前几阶进行合理截断从而降低速度空间计算量。伪谱法是一种有效的计算
  
  谱展开方法的优势:
  以球谐函数作为正交基对速度空间做展开处理,使得算具有一些优势:
  第$l^{th}$角度散射算符只作用于本阶球谐分量,并使得相应幅度以正比于$l(l+1)$的指数速率衰减。若系统各向异性不是很大或者碰撞效应足够强,只需要前几阶球谐函数展开分量足以高精度描述系统演化。
  
\subsubsection{格子玻耳兹曼}
\label{格子玻耳兹曼}

  基于BGK(Bhatnagar-Gross-Krook)近似,得到 Boltzmann-BGK模型

\subsubsection{离散玻耳兹曼}
\label{离散玻耳兹曼}

  (求解连续介质控制方程的粒子方法)
  近平衡态,根据查普曼-恩斯库格(Chapman-Enskog)多尺度分析,
  


  
  
\subsubsection{动理学宏观建模}
\label{动理学宏观建模}

\subsubsection{直接}
\label{直接}



\section{文章结构}

  本文的组织结构如下:

在第2章,基于聚变等离子体的特性,首先对VFP方程进行归一化处理。然后对归一化的VFP方程在速度空间做谱变换,在理论上对其进行简化。Coulomb碰撞在速度空间具有球对称性,因而选择球谐函数为基矢对碰撞项方程在角度方向做谱展开,最后得到其截断的方程组形式。给出谱展开框架下碰撞项的线性化形式。证明非线性化的VFP方程可自洽的退化为常见模型。

在第3章,具体阐述Fokker-Planck碰撞项方程的伪谱算法的构造。我们以双组分碰撞为例,在球谐函数展开基础上,速度轴方向可用Gaussian-Laguerre多项式为基矢展开,当两种组分网格差异性不大时,可用经典伪谱法快速求解。对于更一般的分布函数,速度轴方向以Gaussian-Laguerre格点为初始网格应用自适应网格技术,构造自适应增强伪谱法。最后选取双温热平衡电子-电子碰撞作为算例验证算法的有效性。

在第4章,给出了伪谱算法中离散守恒的构造。我们首先简单介绍了后验分析方法,根据其基本原理,通过做后验映射构造满足离散守恒的伪谱算法。我们分别用双温热平衡电子-电子碰撞以及双温热平衡离子-电子碰撞为例,验证本算法的正确性与长时间守恒性。

在第5章,用守恒的伪谱算法研究快粒子慢化问题。首先针对是否考虑电子韧致辐射,研究燃烧等离子体中高能α粒子在多组分均匀等离子体中的慢化。更一般的,碰撞项方程有更多分量不为零;我们以氘束为例,研究了强流粒子束在均匀背景等离子体中的慢化行为。

在第6章,给出了VFP方程在六维相空间中的有限体积-伪谱算法的构造方法。对于等离子体中更一般的输运过程,空间梯度、平均场效应不能忽略。首先在速度空间做归一化,得到球谐函数展开的VFP方程。然后忽略归一化引入的惯性项,在速度空间采用自适应增强的伪谱方法、在物理空间采用有限体积法构造六维相空间中离散的VFP方程。

在最后的第7章,总结了文中的主要工作,分析了研究的不足之处,展望了这一领域未来的发展趋势。


