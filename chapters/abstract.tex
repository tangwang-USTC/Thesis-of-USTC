% !TeX root = ../main.tex

\ustcsetup{
  keywords = {
    快粒子慢化,Rosenbluth-Fokker-Planck碰撞项,伪谱法,后验分析法
  },
  keywords* = {
    Slowing-down of fast particle, Rosenbluth-Fokker-Planck collision, Pseudo-spectral method, Backward error analysis
  },
}

\begin{abstract}
  在非热平衡的高温等离子体(如燃烧等离子体)中,由于库伦碰撞效应使得等离子体自发地向热平衡态演化,此过程可用Vlasov-Fokker-Planck方程描述。快粒子慢化是燃烧等离子体中的重要物理问题;其高能量特性使得慢化过程是一个具有显著动理学效应的长时间演化的耗散过程。因此,基于Vlasov-Fokker-Planck方程构造满足离散守恒性的有效算法以研究这种长时间演化的动理学行为尤为重要。传统单一数值方法(如有限差分法、有限体积法等)可用于求解六维Vlasov-Fokker-Planck方程,然而通常难以同时满足高效性、鲁棒性与离散守恒性等综合需求。伪谱法与有限体积法是求解偏微分方程的重要数值方法。结合两者分别应用于速度空间与物理空间的数值离散,并基于后验分析方法可方便地构造守恒算法;这是一种能够利用库伦碰撞的球对称性来优化数值求解的方案。本文从Rosenbluth-Fokker-Planck碰撞项方程出发,简要介绍谱分析方法及对Rosenbluth-Fokker-Planck碰撞项方程做谱展开的构造方法、归纳并总结了碰撞项方程的自适应增强伪谱算法、提出了基于后验分析方法的守恒策略、并针对高能α粒子与强流束等快粒子慢化过程做了系统分析,最后扩展到Vlasov-Fokker-Planck方程并结合有限体积法给出六维相空间中的离散格式。
  
对于均匀各向同性等离子体,空间梯度与平均场效应可以忽略,分布函数的演化可用Rosenbluth-Fokker-Planck碰撞项方程(或者等价的Landau碰撞项方程)描述。然而由于它是一个三维高阶偏微分方程,一般而言满足守恒性要求的算法计算量较大,如采用有限体积法直接在速度空间离散。带电粒子之间库伦碰撞在速度空间具有球对称性,且分布函数由于碰撞效应自发趋向于麦克斯韦分布,因此速度空间采用球坐标系并选取合适基矢可以简化碰撞方程。我们首先在速度空间用本地热速度做归一化并获得归一化的碰撞项,然后在速度空间应用谱分析方法,采用球谐函数为基矢对Rosenbluth-Fokker-Planck方程在角度方向展开,最后通过线性化得到碰撞项的简化模型。
  
碰撞项演化方程有多种数值求解方法。我们首先简单回顾线性化碰撞模型的数值求解,然后重点构造自适应增强伪谱法。温度接近的氘-氚碰撞(或者相同质量粒子碰撞)是稳态燃烧等离子体中输运过程的重要微观机制。由于其质量与温度差异较小,因此速度空间网格差异不大。在角度方向用球谐函数展开的基础上,速度轴方向可用Gaussian-Laguerre多项式为基矢展开,此时可用经典伪谱法对Rosenbluth-Fokker-Planck方程进行求解。对于更一般的分布函数情形,我们在角度方向采用经典伪谱法,速度轴方向以Gaussian-Laguerre格点为初始网格并应用自适应网格技术,从而构造自适应增强伪谱法,最后选取双温热平衡电子-电子碰撞作为算例验证算法的有效性。
  
质量、动量与能量守恒是三大基本守恒定律,对应流形理论中的不变量;满足其离散守恒是研究等离子体长时间演化行为的重要基础。然而自适应增强伪谱法只能得到在理论值附近的近似解。后验分析方法是理解长时间演化行为的一个重要分析工具。根据流形映射理论,如果原始算法可以保证计算误差为一个小量,则可以通过在流形面上做后验映射从而构造能够保持守恒性的算法且不改变原始算法的收敛特性。应用以上理论,我们在自适应增强伪谱算法基础上,通过做简单的后验映射构造满足离散守恒的伪谱算法。同样的,我们再次用双温热平衡电子-电子碰撞验证本算法的正确性与长时间守恒性。当两种粒子质量不同时碰撞项中会引入新的分量,因此我们进一步计算了双温热平衡离子-电子碰撞模型。

快粒子慢化是聚变等离子体中的重要物理问题。燃烧等离子体中聚变产生的能量高达几个MeV 的α粒子与加热过程中产生的高能尾巴是快粒子的重要来源。另一方面,聚变等离子体中不同带电粒子碰撞过程具有不同的时间尺度,通常离子-电子碰撞的弛豫时间远大于同种粒子自驰豫碰撞时间。因此,我们首先忽略电子辐射并应用满足离散守恒的自适应增强伪谱算法系统研究了α粒子与电子的碰撞;然后在考虑电子韧致辐射的情况下,分别研究了两组分(α粒子与电子)与多组分(α粒子与电子、氘离子、氚离子)碰撞过程。更一般的,我们以氘束为例,研究了粒子束在均匀背景等离子体中的慢化行为。

对于等离子体中更一般的输运过程,我们采用Vlasov-Fokker-Planck方程同时描述空间梯度、平均场效应与碰撞效应。首先在速度空间做归一化,得到球谐函数展开的Vlasov-Fokker-Planck方程。然后忽略归一化引入的惯性项,在速度空间采用自适应增强的伪谱方法、在物理空间采用有限体积法构造六维相空间中离散的Vlasov-Fokker-Planck方程。

本文阐述的自适应增强伪谱算法实际上是对等离子体分布函数在速度空间的本征坐标系中的最优离散近似。根据谱分析理论可知这种离散化网格具有快速收敛特性并趋近于理论值,因而理论上能够应用后验分析方法方便地构造守恒格式,这是传统方法如有限差分法等难以企及的。自适应增强伪谱算法克服了传统谱分析方法的应用局限性,进一步提高了对任意各向异性度的等离子体的适用性。结合物理空间中采用有限体积法,本文提出的自适应增强伪谱算法有助于我们更准确地模拟和预测等离子体中快粒子的慢化与热化行为,为燃烧等离子体中复杂的输运过程提供新的研究手段。
\end{abstract}

\begin{abstract*}
  In non-thermal equilibrium high-temperature plasma (such as burning plasma) , the spontaneous evolution from non-equilibrium state to thermal equilibrium state due to Coulomb collision which can be described by the Vlasov-Fokker-Planck equation. Slowing-down of fast particle is an important physical problem in burning plasma, and its high energy characteristics make the slowing-down process a long-time evolution process with significant kinetic effects. Therefore, it is very important to enforce an efficient algorithm based on the Vlasov-Fokker-Planck theory to study the dynamical behavior of this kind of long-time evolution. The traditional single numerical method (such as finite difference method, finite-volume method) can be used to solve the six-dimensional Vlasov-Fokker-Planck equation, but it is difficult to satisfy the demands of high efficiency, robustness and discrete conservation at the same time. Pseudo spectral method and finite-volume method are important numerical methods for partial differential equation. Combining the two discretization methods for the velocity space and the physical space respectively, we can construct a conservation algorithm based on the backward error analysis method easily. This will be an optimized scheme which using the spherical symmetry of Coulomb collision. Starting from the Rosenbluth-Fokker-Planck collision equation, this paper introduces the spectral analysis method and the construction method of spectrum expansion for the Rosenbluth-Fokker-Planck collision equation briefly, summarizes the self-adaptive enhanced pseudo-spectral algorithm for the collision equation, proposes a conservation strategy based on a backward error analysis method, and makes a systematic analysis of the fast-particle slowing-down process such as high-energy α and high-current beam. Finally, we extend this algorithm to the Vlasov-Fokker-Planck equation. Combining with the finite-volume method in physics space, we give the discrete scheme in six-dimensional phase space.
  
For homogeneous isotropic plasma, the spatial gradient and the mean-field effect can be neglected, so the evolution of the distribution function will be described by the Rosenbluth-Fokker-Planck collision equation (or the equivalent Landau collision equation). However, because it is a three-dimensional high-order partial differential equation, the computational complexity of the algorithm which guarantees the conservation usually is very large, such as using the finite volume method directly in the velocity space discretion. The Coulomb collision between charged particles is spherically symmetric in the velocity space, and the distribution function tends to Maxwellian distribution spontaneously due to the collision effect, therefore, the collision equation can be simplified by using spherical coordinate system and choosing suitable base vector in velocity space. In this section, we first obtain the normalized collision terms by using the local thermal velocity in the velocity space. Then based on the spectral analysis theory, we expand the Rosenbluth-Fokker-Planck equation in the angular direction using the spherical harmonics. Finally, a simplified model of the collision term is obtained by linearization.

There are many numerical methods to solve the evolution equation of collision term. We briefly review the numerical solution of the linearized collision model, and then focus on constructing the adaptive enhanced pseudo-spectral method. The deuterium-tritium collision (or the same mass particle collision) at near temperature is an important microscopic mechanism of the transport process in steady-state burning plasma. Because the difference between mass and temperature is small, the difference of velocity space grid is not indistinctive. Based on the spherical harmonic expansion, the velocity axis can be expanded by Gaussian-Laguerre polynomials, and the Rosenbluth-Fokker-Planck equation can be solved by the classical pseudo-spectral method. More generally, when distribution function is far from Maxwellian, we use the classical pseudo-spectral method in the angular direction, but an adaptive enhanced pseudo-spectral method in velocity axis direction which is using the Gaussian-Laguerre points as the initial grids and the adaptive grid technique to refine it. Finally, the two-species thermal equilibrium electron-electron collision model is selected as an example to verify the effectiveness of the algorithm.
The conservation of mass, momentum and energy are the three basic conservation laws, which correspond to the invariants in manifold theory. However, the adaptive enhanced pseudo-spectral method can only obtain the approximate solution near the theoretical value. Backward error analysis is an important analytical tool for understanding long-term evolutionary behavior. According to the theory of manifold mapping, if the error of the original algorithm is small, the conservative algorithm can be constructed by posterior mapping on the manifold surface without changing the convergence of the original algorithm. Applying the above theory, we construct a pseudo-spectral algorithm satisfying discrete conservation by a simple posterior mapping based on the adaptive enhanced pseudo-spectral algorithm. In the same way, the correctness and long-time conservation of the proposed algorithm are verified by the two-species thermal equilibrium electron-electron collision. When the masses of two particles are different, a few of new components will be introduced into the collision term, so we further calculate the two-species thermal equilibrium ion-electron collision model.

Slowing-down of fast particle is an important physics problem in fusion plasma. Fusion in a burning plasma produces several MeV α particles and the high-energy tails produced during heating are important sources of fast particles. On the other hand, different charged particle collisions in fusion plasma have different time scales, and generally the relaxation time of ion-electron collisions is much longer than that of the same kind of particles. Therefore, we first ignore the electron radiation and apply the adaptive enhanced pseudo-spectral algorithm to study the α-electron collision. And then considering the electron bremsstrahlung, the collision process between two species (α and electron) and multiple spices (α and electron, Deuterium Ion, tritium ion) is studied respectively. More generally, we study the slowing-down process of deuterium beam in a homogeneous background plasma.

For the more general transport process in plasma, the Vlasov-Fokker-Planck equation is used to describe the spatial gradient, mean field effect and collision effect simultaneously. First, the normalized Vlasov-Fokker-Planck equation with spherical harmonic expansion is obtained by normalizing the velocity space. Then, the discrete Vlasov-Fokker-Planck equation in six-dimensional phase space is constructed by using the adaptive enhanced pseudo-spectral method in the velocity space and the finite-volume method in the physical space.

The adaptive enhanced pseudo-spectral algorithm described in this paper is actually an optimal discrete approximation of the plasma distribution function in the eigen-coordinate system of velocity space. According to the spectral analysis theory, the discretized grid has the characteristics of fast convergence and approaches to the theoretical value, so it is easy to construct the conservative scheme by using a backward error analysis method, which is difficult to achieve by traditional methods such as finite difference method. The adaptive enhanced pseudo-spectral algorithm overcomes the limitations of traditional spectral analysis methods, and further improves the applicability to arbitrary anisotropic plasma. In combination with the use of finite-volume method in physical space, the adaptive enhanced pseudo-spectral algorithm proposed in this paper can help us to simulate and predict the slowing-down and thermalization of fast particles in plasma more accurately. It provides a new research method for the complex transport process in burning plasma.

Applying kinetic model to the scrape-off layer (SOL) plasma, PIC technique is based on the Monte Carlo (MC) techniques for the Coulomb collision effects\cite{Procassini1990}. 

\end{abstract*}
